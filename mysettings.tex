\usepackage[utf8]{inputenc}
\usepackage[margin=0.75in]{geometry}

\usepackage{amsmath,amssymb,amsfonts}
%\usepackage{bbm}

\usepackage{amsthm} %ENV proof
\usepackage{mathtools}
\usepackage{subcaption}
%\usepackage{hhline}

\usepackage{braket}

\usepackage{afterpage}
\usepackage{fancyhdr} 
\usepackage{multicol}
\usepackage{multirow}
\usepackage{framed} 
\usepackage{array} 
\usepackage{graphicx} 
\usepackage{graphics}
\usepackage{mathtools} 
\usepackage{tikz} 
\usepackage{verbatim}
\usepackage{float}
%\usepackage{bm}
\usepackage{comment}
\usepackage{relsize}
\usepackage{xcolor}
\usepackage[Symbol]{upgreek}
\usepackage{placeins}
%\usepackage{esvect}
\usepackage{xspace}
\usepackage{breakcites}
\usepackage{enumerate}
%\usepackage{bbm} % for ones function
\usepackage{setspace}
\usepackage{titlecaps}
\usepackage{makecell}
\usepackage{titlesec}
\usepackage{booktabs} % used to make double clines in tables
\usepackage{colortbl} %used to color rows of a table

\usepackage[]{hyperref}
%\hypersetup{
%	colorlinks
%	%allcolors=black
%	%urlcolor=black
%}
\usepackage{cleveref}
%\cref{section}{S}{SS}
%\Crefname{section}{S}{SS}

\usepackage{wrapfig}

\usepackage[
	lambda,
	advantage,
	operators,
	sets,
	adversary,
	landau,
	probability,
	notions,
	logic,
	ff,
	mm,
	primitives,
	events,
	complexity,
	asymptotics,
	keys
	]{cryptocode}


\let\st\undefined % To avoid conflicts with soul package
\usepackage{soul}

\usepackage[normalem]{ulem} %for the \sout command


% Select what to do with todonotes: 
% \usepackage[disable]{todonotes} % notes not showed
\usepackage[draft]{todonotes}   % notes showed

% to prevent several cites to overflow the margin
\usepackage{breakcites}

% For Chancery font, i.e. \mathpzc
\DeclareMathAlphabet{\mathpzc}{OT1}{pzc}{m}{it}

%\usepackage{bm} %for bold symbol

\usepackage[LGR,T1]{fontenc}

\DeclareUnicodeCharacter{00A0}{ }

\usepackage{adjustbox}
\newcommand{\headrow}[1]{\multicolumn{1}{c}{\adjustbox{angle=45,lap=\width-0.5em}{#1}}}

% = = = Table bullets: \full and \prt (full and part)

\newcommand{\full}{$\bullet$}
\newcommand{\prt}{$\circ$}

\newcommand{\fulltext}{\CIRCLE}
\newcommand{\prttext}{\LEFTcircle}

\newcommand{\fulltable}{$\tiny{\CIRCLE}$}
\newcommand{\prttable}{$\tiny{\LEFTcircle}$}

\usepackage{xargs}                      % Use more than one optional parameter in a new commands
%\newcommandx{\af}[2][1=]{
%%\newcommand{\antonio}[1]{
%\todo[color=orange!50, size=\footnotesize,#1]{{\bf AF}: #2 }
%}

\ifComments
\newcommand{\af}[1]{{\noindent\color{purple!65!black}{\textbf{~#1 -- AF}}}}
\newcommand{\matteo}[1]{{\noindent\color{red!90!black}{\textbf{#1 -- MC}}}}
\newcommand{\dario}[1]{{\noindent\color{blue!90!black}{\textbf{#1 -- DF}}}}
\newcommand{\anais}[1]{{\noindent\color{pink!40!red}{\textbf{#1 -- AQ}}}}
\else
\newcommand{\af}[1]				{}
\newcommand{\matteo}[1]{}
\newcommand{\dario}[1]	{}
\newcommand{\anais}[1]	{}
\fi

%\newcommand{\PSevl}[1][]{\mathcmd{\mathbf{PEvl}\if!#1!\else^{#1}\fi}}

%\def\af{\antonio}

%\newcommand{\afpar}[1]{
%\todo[color=orange!50, size=\footnotesize]{{\bf AF}: #1 }
%}


\newcommand{\tmpdraft}[1]{{\noindent\color{green!25!black}{#1}}}

\newcommand*{\trightarrow}[1]{\xrightarrow{\mathmakebox[5cm]{#1}}}
\newcommand*{\tleftarrow}[1]{\xleftarrow{\mathmakebox[5cm]{#1}}}

\DeclareMathOperator*{\argmax}{arg\,max}
\DeclareMathOperator*{\argmin}{arg\,min}

%\AtBeginEnvironment{align}{\setcounter{equation}{0}}

\usepackage{enumitem}
\setdescription{labelindent=0pt,leftmargin=6pt,itemsep=1pt}
\setitemize{labelindent=2pt,leftmargin=10pt,itemsep=1pt}
\setenumerate{labelindent=1pt,leftmargin=10pt,itemsep=1pt}

%\newenvironment{corollary}{\medskip{\sc{Corollary}} \quad}{}
%\newenvironment{lemma}{\medskip{\sc{Lemma}} \quad}{}
%\newenvironment{proof}{\medskip{\sc{Proof}} \quad}{\hfill$\square$}

\newtheorem{theorem}{Theorem}[section]
\newtheorem{corollary}{Corollary}[theorem]
\newtheorem{lemma}[theorem]{Lemma}
\newtheorem{definition}{Definition}[section]

\titleclass{\subsubsubsection}{straight}[\subsection]

\newcounter{subsubsubsection}[subsubsection]
\renewcommand\thesubsubsubsection{\thesubsubsection.\arabic{subsubsubsection}}
\renewcommand\theparagraph{\thesubsubsubsection.\arabic{paragraph}} % optional; useful if paragraphs are to be numbered

\titleformat{\subsubsubsection}
  {\normalfont\normalsize\scshape}{\thesubsubsubsection}{1em}{}
\titlespacing*{\subsubsubsection}
{0pt}{3.25ex plus 1ex minus .2ex}{1.5ex plus .2ex}

\makeatletter
\renewcommand\paragraph{\@startsection{paragraph}{5}{\z@}%
  {3.25ex \@plus1ex \@minus.2ex}%
  {-1em}%
  {\normalfont\normalsize\bfseries}}
\renewcommand\subparagraph{\@startsection{subparagraph}{6}{\parindent}%
  {3.25ex \@plus1ex \@minus .2ex}%
  {-1em}%
  {\normalfont\normalsize\bfseries}}
\def\toclevel@subsubsubsection{4}
\def\toclevel@paragraph{5}
\def\toclevel@paragraph{6}
\def\l@subsubsubsection{\@dottedtocline{4}{7em}{4em}}
\def\l@paragraph{\@dottedtocline{5}{10em}{5em}}
\def\l@subparagraph{\@dottedtocline{6}{14em}{6em}}
\makeatother

\setcounter{secnumdepth}{4}
\setcounter{tocdepth}{4}
